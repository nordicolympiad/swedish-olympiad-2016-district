\problemname{Rektangelmagi}

En \emph{magisk sekvens} av längd $n + 1$ är en sekvens av tal $a, a + d, a + 2d, ..., a + nd$ för två rationella tal $a$ och $d$, t.ex. $2, 5.5, 9, 12.5$ eller $5, 5, 5$ eller $2, 1, 0, -1, -2$.

En \emph{magisk rektangel} av storlek $R \times C$ är en rektangel där varje rad och kolumn är en magisk sekvens.

Givet en rektangel av heltal där vissa av talen är bortsuddade, avgör om det går att fylla i dessa bortsudda tal så att rektangeln är en magisk rektangel.

\section*{Indata}
Den första raden i indata innehåller talen $R$ och $C$, antalet rader och kolumner i den givna rektangeln. Sedan följer $R$ rader med $C$ heltal vardera.

Ett bortsuddat tal representeras som en punkt.

\section*{Utdata}
Om ingen lösning finns, skriv ut \texttt{ej magisk}.

Annars, skriv ut $R$ rader med $C$ kolumner - en arimetisk rektangel där du tagit indatarektangeln och ersatt de bortsuddade talen.

Rationella tal ska anges på formen \texttt{N/D}, där $N$ och $D$ är högst 100 siffror långa. Observera att det inte ska vara mellanslag mellan talen och divisionstecknet.

Om $D = 1$ kan du skriva \texttt{N}.

\section*{Poäng}

I de första 9 fallen gäller $1 \le R, C \le 6$.

\begin{enumerate}
	\item[10 poäng] alla tal är redan ifyllda.
	\item[10 poäng] Antingen $R$ eller $C$ är 1.
	\item[10 poäng] $R = C = 2$
	\item[10 poäng] varje testfall har en unik lösning, och rektangeln är konstruerad så att det finns en rad eller kolumn med bara ett bortsuddat tal, och när den fylls i finns det återigen en rad eller kolumn med bara ett tal, osv, ända tills hela rektangeln är ifylld. 
	\item[10 poäng] varje testfall har en unik lösning som innehåller enbart heltal.
	\item[10 poäng] varje testfall har en unik lösning.
	\item[10 poäng] varje testfall har antingen en unik lösning som innehåller enbart heltal, eller så har det inte en lösning.
	\item[10 poäng] varje testfall har antingen en unik lösning eller ingen lösning alls.
	\item[10 poäng] inga ytterligare begränsningar.
	\item[10 poäng] inga ytterligare begränsningar, men $1 \le R, C \le 50$.
\end{enumerate}
