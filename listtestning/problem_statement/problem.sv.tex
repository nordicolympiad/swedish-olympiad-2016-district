\problemname{Listtestning}

Mårten har implementerat en \href{https://en.wikipedia.org/wiki/Doubly\_linked\_list}{dubbellänkad lista}. Mårten är nämligen inte så smart. Han vet inte om att det finns färdliga länkade listor i nästan alla standardbibliotek som finns.

Mårten håller inte med dig om att det är dumt - han tycker att sin egna lista är mycket effektivare än den som finns i standardbiblioteket. Det är upp till dig att motbevisa honom, genom att demonstrera att effektivtet inte är allt. Hans lista är nämligen trasig.

Din uppgift är att skriva ett antal testfall som demonstrerat Mårtens trasigheter. Totalt har Mårten gjort 10 försök att skriva en länkad lista, och dina testfall ska skjuta ner så många av Mårtens implementationer som möjligt.

Ett testfall är på följande form:
\begin{description}
	\item[storlek] - fråga vad storleken på listan är.
	\item[pop\_first] - ta bort första elementet i listan.
	\item[pop\_back] - ta bort sista elementet i listan.
	\item[add\_first X] - lägg till heltalet $-1000 \le X \le 1000$ först i listan.
	\item[add\_back X] - lägg till heltalet $-1000 \le X \le 1000$ sist i listan.
	\item[add X Y] - lägg till heltalet $-1000 \le X \le 1000$ på plats $Y$ i listan.
	\item[remove Y] - ta bort elementet på plats $Y$ i listan.
	\item[clear] - ta bort alla element i listan.
\end{description}
Positioner i listan är noll-indexerade. 

Mellan testfall ska du skriva ut en rad med tre bindestreck: \texttt{---}.

\section*{Indata}
Problemet har ingen indata.

\section*{Utdata}
Du ska skriva ut ett antal rader med dina testfall. Du får skriva ut max 1000 rader.

\section*{Poängsättning}
Totalt har Mårten kodat 10 trasiga implementationer. För varje implementation som dina testfall dödar får du 6 poäng.
