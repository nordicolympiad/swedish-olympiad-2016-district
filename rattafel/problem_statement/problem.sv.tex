\problemname{Rätta fel}

Givet en text på engelska, där vissa tecken har blivit ersatta med \texttt{#}, avgör vilka tecken som skulle stått där egentligen.

\section*{Indata}
Indata består av en enda rad med tecken \texttt{a-z, A-Z, 0-9}, skiljetecknen \texttt{.,-:;!?"'}, samt blanksteg.
Ingen testdatafil består av mer än $1\,000\,000$ tecken, inklusive 

\section*{Utdata}
För varje förekomst av \texttt{#} i indatat ska du skriva ut en rad med vilken bokstav som egentligen skulle varit där.

Alla tecken kan ha blivit ersatta, inklusive mellanslag och skiljetecken. Om du ska rätta en bokstav, måste du även ha rätt på stor och liten bokstav.

\section*{Poäng}

Det finns fyra testfall ditt program kommer köras på.
\begin{enumerate}
  \item I ett testfall värt 15 poäng kommer högst en bokstav per ord att ersättas. Inga mellanslag eller skiljetecken kommer ersättas.
  \item I ett testfall värt 25 poäng kommer $10\%$ slumpmässiga tecken bytas ut.
  \item I ett testfall värt 30 poäng kommer $70\%$ slumpmässiga tecken bytas ut.
  \item I ett testfall värt 40 poäng finns inga extra garantier.
\end{enumerate}

Detta poäng kan alltså ge totalt 110 poäng.

Om du på ett visst testfall, som ger $P$ poäng, rättar \texttt{M} fel och det totalt finns \texttt{N} fel får du 
$$P \cdot \frac{M}{N}$$
poäng, avrundat nedåt.

Din totala poäng är summan av poängen för de olika testfallen.
