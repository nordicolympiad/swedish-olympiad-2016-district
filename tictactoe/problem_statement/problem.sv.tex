\problemname{3-i-rad}

Mårten är extremt duktig på 3-i-rad. Så duktig att du aldrig lyckas slå honom! Skriv ett datorprogram som spelar 3-i-rad åt dig, så du kanske har en chans.

Om du inte vet vad 3-i-rad är, kan du läsa hur spelet fungerar \href{https://sv.wikipedia.org/wiki/Tre_i_rad}{på Wikipedia}.

\section*{Indata och utdata}
Detta problem är interaktivt. Du ska först läsa en rad som innehåller antingen \texttt{first} or \texttt{second} - detta säger om du ska dra först eller inte.

Varje gång du ska dra ska du skriva ut ett 3-i-rad-bräde med draget du gjorde, och sedan läsa in ett 3-i-rad-bräde med draget Mårten gjorde.

Om du skriver ut ett bräde där du vinner ska ditt program avslutas. Om du läser in ett bräde där Mårten vunnit ska ditt program avslutas.

Ett bräde ska skrivas ut som 3 rader med 3 tecken vardera - antingen \texttt{.} för tom ruta, \texttt{o} för Mårtens pjäser, eller \texttt{x} för dina pjäser.

\section*{Poäng}

\begin{enumerate}
	\item[13 poäng] du ska spela ett giltigt spel.
	\item[19 poäng] du ska aldrig förlora mot Mårten.
	\item[24 poäng] du ska alltid vinna ett spel om det går.
	\item[14 poäng] du ska aldrig förlora mot Mårten, och du ska alltid vinna ett spel om det går.
\end{enumerate}
